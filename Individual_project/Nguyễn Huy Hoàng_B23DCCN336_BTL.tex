\documentclass{article}
\usepackage[utf8]{inputenc}
\usepackage[vietnamese]{babel}
\usepackage[top=1cm, left=0.8in, right=0.8in]{geometry}
\usepackage{amsfonts, amsmath, amsthm, amssymb}
\usepackage{listings}
\usepackage{color}
\usepackage{graphicx}
\usepackage{caption}
\usepackage{subcaption}
\usepackage{float}
\usepackage{enumitem}
\renewcommand{\lstlistingname}{Code}
\definecolor{codegray}{rgb}{0.95,0.95,0.95}

\lstset{
  backgroundcolor=\color{codegray},
  basicstyle=\ttfamily\footnotesize,
  breaklines=true,
  frame=single,
  captionpos=b,
  tabsize=2,
  language=[x86masm]Assembler,
  morecomment=[l]{;},
  showspaces=false,
  showstringspaces=false
}

\begin{document}
\title{\textbf{Bài tập lớn Kiến trúc máy tính}}
\author{Nguyễn Huy Hoàng - B23DCCN336}
\date{May 2025}

\maketitle
\section{Bài tập cá nhân}

\subsection{Viết chương trình hợp ngữ Assembly}

\subsubsection{Nhập 1 ký tự và in ra màn hình ký tự đó}

\begin{itemize}
\item[--]\textbf{Trạng thái khi chạy chương trình:}

\begin{figure}[H]
    \centering
    \includegraphics[width=0.8\linewidth]{img/bai02.png}
    \caption{Giao diện trước khi nhập ký tự}
    \label{fig:before_input}
\end{figure}

\begin{figure}[H]
    \centering
    \includegraphics[width=0.8\linewidth]{img/bai02(1).png}
    \caption{Giao diện khi đã nhập ký tự}
    \label{fig:after_input}
\end{figure}
\newpage
\item[--]\textbf{Code:}
\begin{lstlisting}[caption={Nhập 1 ký tự và in ra màn hình ký tự đó}]
.MODEL SMALL
.STACK 100H
.DATA
    TB1 DB 'Hay nhap mot ky tu: $'      ; Khai bao TB1
    TB2 DB 13,10,'Ky tu da nhap: $'     ; Khai bao TB2
    OUTPUT DB ?                         ; Khai bao bien KyTu chua co gia tri
.CODE
    MAIN PROC
        ; Khoi tao thanh ghi DS
        MOV AX, @DATA
        MOV DS, AX
    
        ; In ra man hinh xau TB1
        LEA DX, TB1
        MOV AH, 9
        INT 21H
    
        ; Nhap 1 ky tu tu ban phim
        MOV AH, 1
        INT 21H
        MOV OUTPUT, AL
    
        ; In ra man hinh xau TB2
        LEA DX, TB2
        MOV AH, 9
        INT 21H
    
        ; Hien thi ky tu vua nhap
        MOV AH, 2
        MOV DL, OUTPUT
        INT 21H
    
        ; Ket thuc chuong trinh
        MOV AH, 4CH
        INT 21H
    MAIN ENDP
END MAIN
\end{lstlisting}
\end{itemize}

\subsubsection{Đếm chiều dài của một chuỗi ký tự cho trước}

\begin{itemize}
\item[--]\textbf{Trạng thái khi chạy chương trình:}

\begin{figure}[H]
    \centering
    \includegraphics[width=0.8\linewidth]{img/bai10.png}
    \caption{Giao diện khi chạy chương trình}
    \label{fig:before_input}
\end{figure}
\newpage
\item[--]\textbf{Code:}
\begin{lstlisting}[caption={Đếm chiều dài của một chuỗi ký tự cho trước}]
.MODEL SMALL
.STACK 100H
.DATA
    TB1 DB 'Chuoi ban dau la: Kien truc may tinh$'
    TB2 DB 13,10,'Tong chieu dai cua chuoi: $'
    S   DB 'Kien truc may tinh$'
.CODE
    MAIN PROC
        ; Khoi tao cho thanh ghi DS
        MOV AX, @DATA
        MOV DS, AX
    
        ; In ra man hinh xau TB1
        LEA DX, TB1
        MOV AH, 9
        INT 21H
        
        ; Tinh chieu dai chuoi S (dem so ky tu truoc dau $)
        LEA SI, S
        MOV CX, 0   ; CX se dem do dai
    
    CountLoop:
        MOV AL, [SI]
        CMP AL, '$' ; Neu gap ky tu ket thuc thi dung
        JE DoneCounting
        INC CX
        INC SI
        JMP CountLoop
    
    DoneCounting:
        ; In ra man hinh xau TB2
        LEA DX, TB2
        MOV AH, 9
        INT 21H
    
        ; In so ky tu (CX chua do dai chuoi)
        MOV AX, CX
        MOV BX, 10
        XOR CX, CX
    
    PushLoop:
        XOR DX, DX
        DIV BX
        PUSH DX
        INC CX
        CMP AX, 0
        JNZ PushLoop
    
    PrintLoop:
        POP DX
        ADD DL, '0'
        MOV AH, 2
        INT 21H
        LOOP PrintLoop
    
        ; Ket thuc chuong trinh
        MOV AH, 4CH
        INT 21H
    MAIN ENDP
END MAIN
\end{lstlisting}
\end{itemize}
\newpage

\subsubsection{Tìm giá trị lớn nhất và nhỏ nhất của một mảng số}

\begin{itemize}
\item[--]\textbf{Trạng thái khi chạy chương trình:}

\begin{figure}[H]
    \centering
    \includegraphics[width=0.8\linewidth]{img/bai11.png}
    \caption{Giao diện khi chạy chương trình}
    \label{fig:before_input}
\end{figure}

\item[--]\textbf{Code:}
\begin{lstlisting}[caption={Tìm giá trị lớn nhất và nhỏ nhất của một mảng số}]
.Model small
.Stack 100H
.Data
    TB1 DB 'Gia tri lon nhat cua mang: $'
    TB2 DB 13,10,'Gia tri nho nhat cua mang: $'                
    list DB 1,2,3,4,5,6,7,8,0
    MIN DB ?
    MAX DB ?   
.Code
    MAIN Proc
        ; Khoi tao cho thanh ghi DS
        MOV AX, @DATA
        MOV DS, AX
                   
        MOV CX, 9   ; So phan tu trong mang
        LEA SI, list    ; SI tro den dau mang
        MOV AL, [SI]    ; Lay phan tu dau tien
        MOV MAX, AL ; Gan MAX = phan tu dau tien
        MOV MIN, AL ; Gan MIN = phan tu dau tien
        INC SI  ; Tro den phan tu tiep theo
        DEC CX  ; Da xu ly 1 phan tu dau tien nen giam CX
    
    Start:
        LODSB   ; Nap phan tu tiep theo vao AL va tang SI
        CMP AL, MAX
        JLE SkipMax
        MOV MAX, AL
    SkipMax:
        CMP AL, MIN
        JGE SkipMin
        MOV MIN, AL
    SkipMin:
        LOOP Start
    
        ; In TB1
        LEA DX, TB1
        MOV AH, 9
        INT 21H
    
        ; In MAX (chuyen sang ky tu)
        MOV AL, MAX
        ADD AL, '0'
        MOV DL, AL
        MOV AH, 2
        INT 21H


    
        ; In chuoi TB2
        LEA DX, TB2
        MOV AH, 9
        INT 21H
    
        ; In MIN (chuyen sang ky tu)
        MOV AL, MIN
        ADD AL, '0'
        MOV DL, AL
        MOV AH, 2
        INT 21H
    
        ; Ket thuc chuong trinh
        MOV AH, 4CH
        INT 21H
    MAIN Endp
END MAIN
\end{lstlisting}
\end{itemize}

\subsection{Thực hành phân tích khảo sát hệ thống bộ nhớ}

\subsubsection{Khảo sát cấu hình của máy và hệ thống bộ nhớ của máy đang sử dụng (Bộ nhớ trong: ROM, RAM, Cache System, Bộ nhớ ngoài: ổ đĩa cứng, CD, Thiết bị vào ra)}

 \begin{itemize}
    \item[--]\textbf{Sử dụng phần mềm CPU-Z Ver 2.15.0.x64:} CPU Intel Core i5 3740 kiến trúc Ivy Bridge (4 cores/ 4 threads)
\end{itemize}

\begin{figure}[H]
    \centering
    \includegraphics[width=0.6\linewidth]{img/CPU-Z.png}
    \caption{Giao diện CPU-Z}
    \label{fig:before_input}
\end{figure}

\begin{figure}[H]
    \centering
    \begin{subfigure}[b]{0.7\textwidth}
        \centering
        \includegraphics[width=\textwidth]{img/RAM.png}
    \end{subfigure}
    \hfill
    \begin{subfigure}[b]{0.7\textwidth}
        \centering
        \includegraphics[width=\textwidth]{img/RAM(1).png}
    \end{subfigure}
    \caption{Giao diện Ram (gồm 2 thanh RAM 4 GB DDR3 rời có thể nâng cấp)}
    \label{fig:before_input}
\end{figure}

\begin{figure}[H]
    \centering
    \includegraphics[width=0.8\linewidth]{img/Cache.png}
    \caption{Giao diện Cache (7.25MB)}
    \label{fig:before_input}
\end{figure}

\begin{figure}[H]
    \centering
    \includegraphics[width=0.8\linewidth]{img/Device.png}
    \caption{CD, Disk}
    \label{fig:before_input}
\end{figure}

\begin{figure}[H]
    \centering
    \includegraphics[width=0.8\linewidth]{img/Device.png}
    \caption{Thiết bị đầu vào}
    \label{fig:before_input}
\end{figure}


\subsubsection{Dùng công cụ Debug khảo sát nội dung các thanh ghi IP, DS, ES, SS, CS, BP, SP)}

\begin{figure}[H]
    \centering
    \includegraphics[width=0.8\linewidth]{img/debug.png}
    \caption{Dùng Debug khảo sát nội dung các thanh ghi}
    \label{fig:before_input}
\end{figure}

\begin{figure}[H]
    \centering
    \begin{subfigure}[b]{0.7\textwidth}
        \centering
        \includegraphics[width=\textwidth]{img/debug(1).png}
    \end{subfigure}
    \hfill
    \begin{subfigure}[b]{0.7\textwidth}
        \centering
        \includegraphics[width=\textwidth]{img/debug(2).png}
    \end{subfigure}
    \caption{Debug mã nguồn Chương trình 1.1.1 - Nhập 1 ký tự và in ra màn hình ký tự đó}
    \label{fig:before_input}
\end{figure}

\subsubsection{Giải thích nội dung các thanh ghi, trên cơ sở đó giải thích cơ chế quản lý bộ nhớ của hệ thống trong trường hợp cụ thể này(Chương trình 1.1.1 - Nhập 1 ký tự và in ra màn hình ký tự đó)}

\begin{itemize}
    \item[--] \textbf{Phân tích nội dung các thanh ghi từ Debug:}
    \begin{itemize}[label=$+$]
        \item \textbf{IP (Instruction Pointer):} Thanh ghi IP chứa địa chỉ của lệnh tiếp theo sẽ được thực thi. Trong hình Debug, IP bắt đầu từ \texttt{F409:0204} và thay đổi khi chương trình tiến hành (ví dụ: \texttt{F409:021B}, \texttt{F409:0271}). Điều này phản ánh việc CPU trỏ đến các lệnh như \texttt{MOV}, \texttt{INT}, và \texttt{LEA} trong đoạn mã Assembly để thực thi tuần tự, bắt đầu từ địa chỉ \texttt{0100h} (điểm nhập của chương trình).

        \item \textbf{DS (Data Segment):} Thanh ghi DS chứa địa chỉ của đoạn dữ liệu, được khởi tạo bằng \texttt{MOV AX, @DATA} và sau đó gán vào DS. Trong debug, DS có giá trị như \texttt{0720h}, trỏ đến khu vực bộ nhớ chứa các biến \texttt{TB1}, \texttt{TB2}, và \texttt{OUTPUT}. Giá trị này cho phép chương trình truy cập dữ liệu tĩnh.

        \item \textbf{ES (Extra Segment):} Thanh ghi ES thường được sử dụng cho các thao tác đặc biệt (như di chuyển chuỗi), nhưng trong chương trình này, ES không được sử dụng trực tiếp. Giá trị ES (\texttt{0790h} trong debug) có thể giống DS nếu không có thay đổi, cho thấy nó không tham gia vào logic chính.

        \item \textbf{SS (Stack Segment):} Thanh ghi SS chứa địa chỉ của đoạn stack, được thiết lập mặc định khi chương trình khởi động với \texttt{.STACK 100H} (256 byte). Trong debug, SS là \texttt{0710h}, và SP (\texttt{09FAh}) trỏ đến đỉnh stack, nơi lưu trữ các giá trị tạm thời (như trong các lệnh \texttt{PUSH} hoặc \texttt{POP} nếu có).

        \item \textbf{CS (Code Segment):} Thanh ghi CS chứa địa chỉ của đoạn mã, nơi lưu trữ các lệnh thực thi. Trong debug, CS là \texttt{0723h}, và kết hợp với IP, nó trỏ đến khu vực mã lệnh như \texttt{MOV AX, @DATA} hoặc \texttt{INT 21H}.

        \item \textbf{BP (Base Pointer):} Thanh ghi BP thường dùng làm con trỏ cơ sở cho stack frame, nhưng trong chương trình này, nó không được sử dụng rõ ràng. Giá trị BP (\texttt{0000h} trong một số bước) có thể là mặc định, không thay đổi trong quá trình thực thi.

        \item \textbf{SP (Stack Pointer):} Thanh ghi SP trỏ đến đỉnh của stack. Trong debug, SP thay đổi (ví dụ: từ \texttt{09FAh} đến \texttt{B401h}), phản ánh việc stack được sử dụng để lưu trữ tạm thời khi thực thi các lệnh như \texttt{INT 21H} (gọi hàm BIOS).

        \item \textbf{AX, BX, CX, DX (Thanh ghi tổng quát):} \texttt{AX} chứa dữ liệu tạm thời, ví dụ \texttt{0720h} khi gán DS, hoặc \texttt{0000h} sau khi khởi tạo. \texttt{DX} được dùng để lưu offset của \texttt{TB1} và \texttt{TB2} (ví dụ: \texttt{D21h} khi \texttt{LEA DX, TB1}). \texttt{CX} và \texttt{DX} cũng được sử dụng trong các lệnh delay (nếu có trong chương trình khác).

        \item \textbf{Flags (NU, UP, EI, PL, NZ, NA, PO, NC):} Các cờ trạng thái (flags) như NU (No Unordered), UP (Underflow Pending), EI (Interrupt Enable), PL (Positive), NZ (Not Zero), NA (No Align Check), PO (Parity Odd), NC (No Carry) phản ánh trạng thái của CPU sau mỗi lệnh. Ví dụ, cờ EI cho thấy ngắt được bật, cho phép \texttt{INT 21H} hoạt động.
    \end{itemize}
    
    \item[--] \textbf{Cơ chế quản lý bộ nhớ của hệ thống:}
    \begin{itemize}[label=$+$]
        \item Hệ thống sử dụng mô hình \textbf{segmented memory} (bộ nhớ phân đoạn) đặc trưng của kiến trúc x86 thực thi ở chế độ thực (real mode). Bộ nhớ được chia thành các đoạn (CS, DS, SS, ES), mỗi đoạn có kích thước tối đa 64KB. Địa chỉ logic được tính bằng cách nhân giá trị đoạn với 16 (shift left 4 bit) và cộng với offset (ví dụ: DS:SI hoặc CS:IP).

        \item \textbf{Phân chia bộ nhớ:}
        \begin{itemize}
            \item \textbf{Code Segment (CS):} Chứa mã lệnh Assembly (từ \texttt{MAIN PROC} đến \texttt{END MAIN}), được trỏ bởi CS:IP. Giá trị CS (\texttt{0723h}) kết hợp với IP (\texttt{0204h}, \texttt{021Bh}, v.v.) xác định vị trí lệnh hiện tại.
            \item \textbf{Data Segment (DS):} Chứa dữ liệu tĩnh như \texttt{TB1}, \texttt{TB2}, và \texttt{OUTPUT}, được truy cập qua DS và offset (ví dụ: \texttt{LEA DX, TB1} với DX = \texttt{D21h}). Giá trị DS (\texttt{0720h}) trỏ đến khu vực dữ liệu.
            \item \textbf{Stack Segment (SS):} Dùng để lưu trữ stack, quản lý bởi SS:SP. Với \texttt{.STACK 100H}, stack có kích thước 256 byte, và SP (\texttt{09FAh}) thay đổi khi gọi hàm (như \texttt{INT 21H}).
        \end{itemize}
    \end{itemize}
\end{itemize}

\newpage
\section{Bài tập nhóm: Traffic light system}

\begin{figure}[H]
    \centering
    \includegraphics[scale = 0.7]{img/Traffic_light_system.png}
    \caption{Hình ảnh giao diện khi chạy chương trình}
    \label{fig:before_input}
\end{figure}

\begin{lstlisting}[caption={Traffic light system}]
#start=Traffic_Lights.exe#

name "traffic"
.Model Small
.Stack 100h
.Data
    ; FEDC_BA98_7654_3210
    situation dw 0000_0011_0000_1100b ; Xanh doc  
    s1 dw 0000_0010_1000_1010b  ; Vang doc
    s2 dw 0000_1000_0110_0001b  ; Xanh ngang
    s3 dw 0000_0100_0101_0001b  ; Vang ngang
    sit_end = $

    all_red equ 0000_0010_0100_1001b
.Code
MAIN Proc
     MOV AX, @DATA
    MOV DS, AX
    
    MOV AX, all_red
    OUT 4, AX
    
    MOV SI, offset situation

next:
    MOV AX, [SI]
    OUT 4, AX

    MOV CX, 0FH ; 004C4B40h = 5,000,000
    MOV DX, 4240H
    MOV AH, 86H
    INT 15H   
    
    MOV CX, 0FH ; 004C4B40h = 5,000,000
    MOV DX, 4240H
    MOV AH, 86H
    INT 15H
    
    ADD SI, 2   ; next situation
    CMP SI, sit_end
    JB  next
    MOV SI, offset situation
    JMP next
MAIN Endp
End MAIN
\end{lstlisting}

\end{document}
